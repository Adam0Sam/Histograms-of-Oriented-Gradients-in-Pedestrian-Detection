\section{Conclusions}

This investigation evaluated 864 different sets of HOG parameters to determine the optimal configuration for a linear SVM pedestrian classifier.

Results revealed universally effective parameter values: (2,2) block sizes and (8,8) cell sizes consistently provided optimal local-global feature balance, while preserving gradient information at window boundaries sizeably improves classifier performance without significant overhead. Increased block overlap generally enhances performance, though smaller block sizes seem to benefit less from this effect. Orientation bin size, on the other hand, as it increases, exhibits greatly diminishing returns and increased computational complexity, suggesting the need to balance a system's capacity to handle high-dimensional spaces and the requirement for marginally better accuracy.

The choice of window size proves to be incredibly dataset-specific, with the PnPLO dataset benefiting from smaller window sizes that focus on local features, while the INRIA and Caltech datasets perform better with larger window sizes that capture more contextual information. This suggests that the choice of window size should be tailored to the specific needs of the application, a roadside security system might benefit from larger windows, while a museum inventory system might prefer to differentiate between mannequins and visitors with smaller windows.

Hopefully, this investigation provides valuable insights into the trade-offs and considerations that must be made when selecting HOG parameters for pedestrian detection. Future work could explore the impact of these parameters on other object detection tasks, such as vehicle detection or facial recognition, to determine if the trends observed in this study are consistent across different object classes and datasets.

