\section{Conclusions}

In this investigation, 864 different sets of HOG parameters were evaluated to determine the optimal configuration for a linear SVM pedestrian classifier.   

The results show that there are a number of HOG parameter values which appear to be universally efficient across a wide range of datasets. Block sizes of (2,2) and cell sizes of (8,8) consistently provide the best trade-off between local features and global context. And the use of a "holistic" derivative mask, which seeks to preserve gradient information at object boundaries, significantly improves performance, particularly for low-resolution datasets.

Parameters like orientation bin size and block stride values, as they increase, exhibit diminishing performance returns and increasing vector dimensionality, suggesting the need to balance a system's capacity to handle higher-dimensional spaces and the requirement for marginally better accuracy.

The choice of window size proves to be incredibly dataset-specific, with the PnPLO dataset benefiting from smaller window sizes that focus on local features, while the INRIA and Caltech datasets perform better with larger window sizes that capture more contextual information. This suggests that the choice of window size should be tailored to the specific needs of the application, a roadside security system might benefit from larger windows, while a museum inventory system might prefer smaller windows to differentiate between mannequins and visitors.

Hopefully, this paper provides valuable insights into the trade-offs and considerations that must be made when selecting HOG parameters for pedestrian detection systems. Future work could explore the impact of these parameters on other object detection tasks, such as vehicle detection or facial recognition, to determine if the trends observed in this study are consistent across different object classes and datasets.

