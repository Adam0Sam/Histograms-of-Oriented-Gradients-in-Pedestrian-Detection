\appendix
\section{Appendices}

\subsection{Additional Explanations}

\subsection{Defining Matthew's Correlation Coefficient}\label{appendix:matthews_correlation}
$$\mathrm{MCC} = \frac{\mathrm{TP}\cdot\mathrm{TN}-\mathrm{FP}\cdot\mathrm{FN}}{\sqrt{(\mathrm{TP}+\mathrm{FP})\cdot(\mathrm{TP}+\mathrm{FN})\cdot(\mathrm{TN}+\mathrm{FP})\cdot(\mathrm{TN})+\mathrm{FN}}}$$ 
The values of MCC are bounded within the range $[-1;1]$, where 1 represents a perfect prediction, 0 represents random prediction and -1 total disagreement between prediction and observation. Refer to \cite{chicco_jurman_2020_mcc_f1} regarding the necessary normalization to make the MCC values bounded within $[0;1]$ so that they can be plotted against F1 scores (which themselves are bounded in $[0;1]$)

\subsubsection{Deriving the Frames Per Person in the Caltech Dataset}\label{appendix:caltech_frames_per_person}
The total running time of the Caltech videos is $\sim 10\mathrm{h}$ \cite{dollar_2009_pedestrian}, this gives a frame per second rate of $\sim 7 \text{ frames}/\mathrm{s}$. Since a person is present in a video for $\sim 5 \mathrm{s}$ \cite{dollar_2009_pedestrian}, we can approximate that each identifiable individual will, on average, be present in $34 \text{ frames}$
\subsubsection{Choice of Grayscale Conversion}\label{appendix:grayscale_discuss}
The primary motivation behind the simple color mapping in equation \ref{eq:colour_map} is the wide adoption in image processing libraries, such as \href{https://scikit-image.org/}{scikit-image}. While there are more sophisticated methods of RGB to grayscale conversion \cite{madk_2008_perceptual}, the computational overhead is difficult to justify given the the rather ambiguous nature of assessing which specific color space transformation produces universally desirable outputs for all involved input images \cite{madk_2008_perceptual}.
\input{sections/appendices/code-snippets.tex}
\subsection{Tables of Score Data}\label{appendix:tables_of_data}
\newpage
\subsubsection{INRIA Score Tables}

\subsubsection*{(100, 50) Window}

\begin{table}
    \caption{INRIA Results - (100, 50) Window}
    \includegraphics[width=\linewidth]{/Users/adamsam/repos/ee/Pedestrian-Detection/code/paper/tables/INRIA/INRIA_(100, 50)_0.png}
    \label{tab:INRIA_(100, 50)_0}
\end{table}

\begin{table}
    \caption{INRIA Results - (100, 50) Window}
    \includegraphics[width=\linewidth]{/Users/adamsam/repos/ee/Pedestrian-Detection/code/paper/tables/INRIA/INRIA_(100, 50)_120.png}
    \label{tab:INRIA_(100, 50)_120}
\end{table}

\begin{table}
    \caption{INRIA Results - (100, 50) Window}
    \includegraphics[width=\linewidth]{/Users/adamsam/repos/ee/Pedestrian-Detection/code/paper/tables/INRIA/INRIA_(100, 50)_160.png}
    \label{tab:INRIA_(100, 50)_160}
\end{table}

\begin{table}
    \caption{INRIA Results - (100, 50) Window}
    \includegraphics[width=\linewidth]{/Users/adamsam/repos/ee/Pedestrian-Detection/code/paper/tables/INRIA/INRIA_(100, 50)_200.png}
    \label{tab:INRIA_(100, 50)_200}
\end{table}

\begin{table}
    \caption{INRIA Results - (100, 50) Window}
    \includegraphics[width=\linewidth]{/Users/adamsam/repos/ee/Pedestrian-Detection/code/paper/tables/INRIA/INRIA_(100, 50)_40.png}
    \label{tab:INRIA_(100, 50)_40}
\end{table}

\begin{table}
    \caption{INRIA Results - (100, 50) Window}
    \includegraphics[width=\linewidth]{/Users/adamsam/repos/ee/Pedestrian-Detection/code/paper/tables/INRIA/INRIA_(100, 50)_80.png}
    \label{tab:INRIA_(100, 50)_80}
\end{table}

\subsubsection*{(112, 48) Window}

\begin{table}
    \caption{INRIA Results - (112, 48) Window}
    \includegraphics[width=\linewidth]{/Users/adamsam/repos/ee/Pedestrian-Detection/code/paper/tables/INRIA/INRIA_(112, 48)_0.png}
    \label{tab:INRIA_(112, 48)_0}
\end{table}

\begin{table}
    \caption{INRIA Results - (112, 48) Window}
    \includegraphics[width=\linewidth]{/Users/adamsam/repos/ee/Pedestrian-Detection/code/paper/tables/INRIA/INRIA_(112, 48)_120.png}
    \label{tab:INRIA_(112, 48)_120}
\end{table}

\begin{table}
    \caption{INRIA Results - (112, 48) Window}
    \includegraphics[width=\linewidth]{/Users/adamsam/repos/ee/Pedestrian-Detection/code/paper/tables/INRIA/INRIA_(112, 48)_160.png}
    \label{tab:INRIA_(112, 48)_160}
\end{table}

\begin{table}
    \caption{INRIA Results - (112, 48) Window}
    \includegraphics[width=\linewidth]{/Users/adamsam/repos/ee/Pedestrian-Detection/code/paper/tables/INRIA/INRIA_(112, 48)_200.png}
    \label{tab:INRIA_(112, 48)_200}
\end{table}

\begin{table}
    \caption{INRIA Results - (112, 48) Window}
    \includegraphics[width=\linewidth]{/Users/adamsam/repos/ee/Pedestrian-Detection/code/paper/tables/INRIA/INRIA_(112, 48)_40.png}
    \label{tab:INRIA_(112, 48)_40}
\end{table}

\begin{table}
    \caption{INRIA Results - (112, 48) Window}
    \includegraphics[width=\linewidth]{/Users/adamsam/repos/ee/Pedestrian-Detection/code/paper/tables/INRIA/INRIA_(112, 48)_80.png}
    \label{tab:INRIA_(112, 48)_80}
\end{table}

\subsubsection*{(128, 64) Window}

\begin{table}
    \caption{INRIA Results - (128, 64) Window}
    \includegraphics[width=\linewidth]{/Users/adamsam/repos/ee/Pedestrian-Detection/code/paper/tables/INRIA/INRIA_(128, 64)_0.png}
    \label{tab:INRIA_(128, 64)_0}
\end{table}

\begin{table}
    \caption{INRIA Results - (128, 64) Window}
    \includegraphics[width=\linewidth]{/Users/adamsam/repos/ee/Pedestrian-Detection/code/paper/tables/INRIA/INRIA_(128, 64)_120.png}
    \label{tab:INRIA_(128, 64)_120}
\end{table}

\begin{table}
    \caption{INRIA Results - (128, 64) Window}
    \includegraphics[width=\linewidth]{/Users/adamsam/repos/ee/Pedestrian-Detection/code/paper/tables/INRIA/INRIA_(128, 64)_160.png}
    \label{tab:INRIA_(128, 64)_160}
\end{table}

\begin{table}
    \caption{INRIA Results - (128, 64) Window}
    \includegraphics[width=\linewidth]{/Users/adamsam/repos/ee/Pedestrian-Detection/code/paper/tables/INRIA/INRIA_(128, 64)_200.png}
    \label{tab:INRIA_(128, 64)_200}
\end{table}

\begin{table}
    \caption{INRIA Results - (128, 64) Window}
    \includegraphics[width=\linewidth]{/Users/adamsam/repos/ee/Pedestrian-Detection/code/paper/tables/INRIA/INRIA_(128, 64)_40.png}
    \label{tab:INRIA_(128, 64)_40}
\end{table}

\begin{table}
    \caption{INRIA Results - (128, 64) Window}
    \includegraphics[width=\linewidth]{/Users/adamsam/repos/ee/Pedestrian-Detection/code/paper/tables/INRIA/INRIA_(128, 64)_80.png}
    \label{tab:INRIA_(128, 64)_80}
\end{table}

\subsubsection*{(128, 96) Window}

\begin{table}
    \caption{INRIA Results - (128, 96) Window}
    \includegraphics[width=\linewidth]{/Users/adamsam/repos/ee/Pedestrian-Detection/code/paper/tables/INRIA/INRIA_(128, 96)_0.png}
    \label{tab:INRIA_(128, 96)_0}
\end{table}

\begin{table}
    \caption{INRIA Results - (128, 96) Window}
    \includegraphics[width=\linewidth]{/Users/adamsam/repos/ee/Pedestrian-Detection/code/paper/tables/INRIA/INRIA_(128, 96)_120.png}
    \label{tab:INRIA_(128, 96)_120}
\end{table}

\begin{table}
    \caption{INRIA Results - (128, 96) Window}
    \includegraphics[width=\linewidth]{/Users/adamsam/repos/ee/Pedestrian-Detection/code/paper/tables/INRIA/INRIA_(128, 96)_160.png}
    \label{tab:INRIA_(128, 96)_160}
\end{table}

\begin{table}
    \caption{INRIA Results - (128, 96) Window}
    \includegraphics[width=\linewidth]{/Users/adamsam/repos/ee/Pedestrian-Detection/code/paper/tables/INRIA/INRIA_(128, 96)_200.png}
    \label{tab:INRIA_(128, 96)_200}
\end{table}

\begin{table}
    \caption{INRIA Results - (128, 96) Window}
    \includegraphics[width=\linewidth]{/Users/adamsam/repos/ee/Pedestrian-Detection/code/paper/tables/INRIA/INRIA_(128, 96)_40.png}
    \label{tab:INRIA_(128, 96)_40}
\end{table}

\begin{table}
    \caption{INRIA Results - (128, 96) Window}
    \includegraphics[width=\linewidth]{/Users/adamsam/repos/ee/Pedestrian-Detection/code/paper/tables/INRIA/INRIA_(128, 96)_80.png}
    \label{tab:INRIA_(128, 96)_80}
\end{table}

\subsubsection{Caltech Score Tables}
\input{sections/appendices/caltech_30-tables.tex}
\subsubsection{PnPLO Score Tables}
\input{sections/appendices/PnPLO-tables.tex}
\input{}