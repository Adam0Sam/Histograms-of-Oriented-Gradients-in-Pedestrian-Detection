\section{Introduction}\label{sec:introduction}
Pedestrian detection is one of the most extensively studied fields in computer vision \cite{dollar_2012_pedestrian}, with applications spanning autonomous vehicles, surveillance systems, and robotics \cite{dollar_2012_pedestrian}. Most notably, automatically detecting pedestrians from moving vehicles could considerable impact economic and social welfare by reducing pedestrian injuries and fatalities, which, in the European Union, make up 20\% of all road accidents \cite{slootmans_2021_european}.

Identifying human figures in images or video frames presents unique challenges due to the variability in occlusions, diverse backgrounds and changing environmental conditions \cite{dollar_2012_pedestrian}. Alongside the many variables present in natural pedestrian environments, due to imperfect acquisition instruments, noise can be introduced into images \cite{faraji_2006_ccd}, further highlighting the need of a feature set which would be able to characterize object shape and orientation locally, so that changes in “noisy” regions (like an image’s background) do not significantly impact the feature detection in other regions that still provide useful information (like a pedestrian’s silhouette).

Introduced by Dalal and Triggs \cite{dalal_2005_histograms}, Histograms of Oriented Gradients (HOG) is a well-known \cite{dollar_2012_pedestrian} image processing algorithm because it solves the problem of variability and noise in pedestrian images by detecting one of the most essential features of images — edges \cite{niebles2012edge} \cite{dalal_2005_histograms}. Despite the suggested superiority of HOG \cite{dalal_2005_histograms} and widespread adoption in modern pedestrian classifiers \cite{dollar_2012_pedestrian}, the parameters used for the algorithm have remained essentially unchanged since the introduction of the method in 2005 \cite{dalal_2005_histograms}

Given the importance of the HOG descriptor in real life applications and significant improvements in the variety, difficulty and scale of pedestrian datasets since 2005 \cite{dollar_2012_pedestrian}, this investigation seeks to maximize the performance of a linear Support Vector Machine (SVM) in pedestrian classification by varying the parameters of HOG.