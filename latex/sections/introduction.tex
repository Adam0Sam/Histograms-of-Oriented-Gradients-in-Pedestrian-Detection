\section{Introduction}\label{sec:introduction}
Pedestrian detection is a critical area of research in computer vision and artificial intelligence, with it being one of the extensively studied fields in the past decade \cite{dollar_2012_pedestrian}. The applications of automatic pedestrian detection span autonomous vehicles, surveillance systems, and robotics \cite{dollar_2012_pedestrian}. Most notably, automatically detecting pedestrians from moving vehicles could considerable impact economic and social welfare by substantially reducing pedestrian injuries and fatalities, which, in the European Union, make up 20\% of all road accidents \cite{slootmans_2021_european}

Pedestrian detection involves identifying and locating human figures in images or video frames, which presents unique challenges due to the variability in occlusions, diverse backgrounds and changing environmental conditions \cite{dollar_2012_pedestrian}. Alongside the many variables present in natural pedestrian environments, noise can arise during the image acquisition process, mainly due to imperfect instruments \cite{faraji_2006_ccd}. Therefore cleanly discriminating human appearance and the wide range of poses they can adopt calls for the use of a feature set which would be able to characterise object shape and orientation locally, so that changes in “noisy” regions (like an image’s background) do not significantly impact the feature detection in other regions that still provide useful information (like a pedestrian’s silhouette).

Histograms of Oriented Gradients (HOG) \cite{dalal_2005_histograms} is well-known \cite{dollar_2012_pedestrian} image processing algorithms because it solves the problem of variability and noise in pedestrian images by detecting one of the most essential features of images - edges \cite{niebles2012edge} \cite{dalal_2005_histograms}. Despite the suggested superiority of HOG \cite{dalal_2005_histograms} and widespread adoption in modern pedestrian classifiers \cite{dollar_2012_pedestrian}, the parameters used for the algorithm have remained essentially unchanged since the introduction of the method in 2005 \cite{dalal_2005_histograms}

Given the importance of the HOG descriptor in real life applications and significant improvements in the variety, difficulty and scale of pedestrian datasets since 2005 \cite{dollar_2012_pedestrian}, this investigation seeks to maximize the accuracy and performance of a linear Support Vector Machine (SVM) in pedestrian classification by varying the various properties of HOG.